\chapter{Создание кластерной системы}

В этой главе рассматриваются пошаговые действия при подготовке
автоматизированной установки кластерной системы из 3-ех нод,
описанной в предыдущей главе.

\section{Создание виртуальных сетей}

Для создания новой виртуальной сети в \textmd{virt-manager}
перейдем в "Правка"$\rightarrow$"Свойства подключения"$\rightarrow$"Виртуальные сети"
и в открывшемся диалоговом окне найдем кнопку по созданию сети.
Конфигурации сетей изображены на Рис. \ref{network-mpi} и
\ref{network-mpi-offline}.

\myImage{Конфигурация первой сети для доступа в Интернет}{network-mpi}{network-mpi}
\myImage{Конфигурация второй, внутренней сети}{network-mpi-offline}{network-mpi-offline}

\clearpage

\section{Получение основного kickstart-файла}

Следующим шагом, мы должны создать kickstart-файл, где будут
описаны все основные этапы установки системы. Kickstart-файл
представляет из себя простой текстовый файл (plain text)
который разбит на обязательные и опциональные секции. Создание
и отладка такого файла -- довольно трудоемкая задача. Для
упрощения его получения, можно провести тестовую установку
системы, после чего в домашней папке пользователя \textmd{root}
будет лежать файл \textmd{anaconda-ks.cfg}, сохронивший в
себе все этапы выбора при установке.

\myImage{Добавление второго сетевого интерфейса}{centos-vm}{centos-vm}

Для этого выберем пункт "Создать виртуальную машину" и
проследуем мастеру настройки, указав скаченный с официального сайта
образ CentOS и созданную сеть с доступом в Интернет. Также выберем пункт
"Дополнить конфигурацию перед установкой", чтобы добавить
второй интерфейс к доступу в изолированную сетью (Рис\ref{centos-vm}),
модель интерфейсов \textmd{virtio}, а для ускорения установки
в параметрах жесткого диска можно указать "unsafe" - кэширования.
\clearpage

\myImage{Процесс установки CentOS7}{centos-install}{centos-install}

После настройки физической конфигурации машины, мы можем нажать
"Начать установку" и перед нами раскроется окно с эмуляцией
монитора созданной машины. В ней мы увидим приглашение к установке
CentOS7, в которой можем произвести всю необходимую настройку
(Рис. \ref{centos-install}). После того, как мы
нажмем "Начать установку", мы сможем настроить пользователей
системы.

Наконец, перезагрузив систему и зайдя от \textmd{root}-пользователя
мы найдем файл \textmd{anaconda-ks.cfg}. При желании полученный
файл можно найти в приложении \ref{app:kickstart}.

\clearpage
\section{Генерация kickstart-файла с помощью python}


