\chapter{Создание кластерной системы}

В этой главе рассматриваются пошаговые действия при подготовке
автоматизированной установки кластерной системы из 3-ех нод,
описанной в предыдущей главе.

\section{Создание виртуальных сетей}

Для создания новой виртуальной сети в \textmd{virt-manager}
перейдем в "Правка"$\rightarrow$"Свойства подключения"$\rightarrow$"Виртуальные сети"
и в открывшемся диалоговом окне найдем кнопку по созданию сети.
Конфигурации сетей изображены на Рис. \ref{network-mpi} и
\ref{network-mpi-offline}.

\myImage{Конфигурация первой сети для доступа в Интернет}{network-mpi}{network-mpi}
\myImage{Конфигурация второй, внутренней сети}{network-mpi-offline}{network-mpi-offline}

\clearpage

\section{Получение основного kickstart-файла}

Следующим шагом, мы должны создать kickstart-файл, где будут
описаны все основные этапы установки системы. Kickstart-файл
представляет из себя простой текстовый файл (plain text)
который разбит на обязательные и опциональные секции. Создание
и отладка такого файла -- довольно трудоемкая задача. Для
упрощения его получения, можно провести тестовую установку
системы, после чего в домашней папке пользователя \textmd{root}
будет лежать файл \textmd{anaconda-ks.cfg}, сохронивший в
себе все этапы выбора при установке.

\myImage{Добавление второго сетевого интерфейса}{centos-vm}{centos-vm}

Для этого выберем пункт "Создать виртуальную машину" и
проследуем мастеру настройки, указав скаченный с официального сайта
образ CentOS и созданную сеть с доступом в Интернет. Также выберем пункт
"Дополнить конфигурацию перед установкой", чтобы добавить
второй интерфейс к доступу в изолированную сетью (Рис\ref{centos-vm}),
модель интерфейсов \textmd{virtio}, а для ускорения установки
в параметрах жесткого диска можно указать "unsafe" - кэширования.
\clearpage

\myImage{Процесс установки CentOS7}{centos-install}{centos-install}

После настройки физической конфигурации машины, мы можем нажать
"Начать установку" и перед нами раскроется окно с эмуляцией
монитора созданной машины. В ней мы увидим приглашение к установке
CentOS7, в которой можем произвести всю необходимую настройку
(Рис. \ref{centos-install}). После того, как мы
нажмем "Начать установку", мы сможем настроить пользователей
системы.

Наконец, перезагрузив систему и зайдя от \textmd{root}-пользователя
мы найдем файл \textmd{anaconda-ks.cfg}. При желании полученный
файл можно найти в приложении \ref{app:kickstart}.

\clearpage

\section{Генерация kickstart-файла с помощью python}
Для установки нескольких узлов, нам потребуется создать
копии полученного \textmd{kickstart}-файла. При этом
данные файлы будут отличаться только именем узла и
статическим IP-адрессом в изолированной сети. Мы можем
создать из полученного файла Jinja-шаблон\cite{jinja2}, и
сгенерировать полученные конфиги автоматически, в зависимости
от номера узла:

\verbatiminput{listings/generate.py}

А в шаблоне заменить изменяющиеся параметры:

\begin{verbatim}
# ...

network  --bootproto=static --device=eth1 --ip={{ip}} \
           --netmask=255.255.255.0 --ipv6=auto --activate
network  --hostname={{host}}

# ...
\end{verbatim}

После чего мы можем просто раздать http-сервером эти файлы.
И при установке дописав параметр, который запустит
автоматическую установку:
\begin{verbatim}
  inst.ks=http://192.168.127.1:8080/node1.cfg
\end{verbatim}

\section{Настройка NFS}

Рассмотрим настройку NFS-подсистемы, которая сделает общий
подкаталог \textmd{/nfs} для всех узлов. Данный критерий не
обязателен при работе с MPI, но очень упрощает разработку.

\subsection{Настройка NFS-сервера}

Для настройки nfs в CentOS есть пакеты \textmd{nfs-utils} и
\textmd{nfs-utils-lib}. В процессе установки
использовался DVD-образ CentOS7, и эти пакеты были выбраны
при установке. Поэтому остается только провести настройку
узлов:

\begin{verbatim}
mkdir -p /nfs # создадим общую
vim /etc/exports # правим конфигурацию nfs-сервера
\end{verbatim}

В /etc/exports вписываем следующую строку:
\begin{verbatim}
/nfs 10.20.30.0/24(rw,sync,no_root_squash,no_all_squash)
\end{verbatim}

При этом:
\begin{itemize}
  \item /home/nfs – расшариваемая директория;
  \item 10.20.30.40/24 – IP адрес клиента (или, как в моем случае, возможность подключения для всей подсети);
  \item rw – разрешение на запись;
  \item sync – синхронизация указанной директории;
  \item no\_root\_squash – включение root привилегий;
  \item no\_all\_squash — включение пользовательской авторизации;
\end{itemize}

После чего включаем все необходимые сервисы:
\begin{verbatim}
systemctl enable rpcbind
systemctl enable nfs-server
systemctl enable nfs-lock
systemctl enable nfs-idmap
systemctl start rpcbind
systemctl start nfs-server
systemctl start nfs-lock
systemctl start nfs-idmap
\end{verbatim}

\subsection{Настройка NFS-клиента}



\section{Установка Intel MPI}

\section{Запуск тестовой программы}
