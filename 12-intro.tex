\Introduction

Задача данного проекта по модулю возникла в рамках соревнования по 
высокопроизводительному программированию Asia Supercomputer Community (ASC)
\cite{asc}. В рамках этого соревнования требуется собрать и настроить 
кластерную систему, после чего решить с ее использованием ряда задач:
тестирование полученной системы на производительность; оптимизации решений,
предоставленных организаторами и решение прикладных задач с нуля.

Чтобы уделить решению задач наибольшее время нужно минимизировать, на сколько
это возможно, время настройки кластерной системы. В момент написания
данного проекта пока не известна вся конфигурация кластерной системы, 
поэтому основной задачей является автоматическое разворачивание $n$-узловой
кластерной системы на виртуальных машинах с операционной системой 
\textmd{CentOS7} и установленной MPI-системой. Коректность работы MPI будем проверять
запуском тестовых програм.

Число узлов $n$ в работе подразумеваем не большим (не более 10--20). Хотя вполне 
возможно, что полученное решение будет позволять развернуть большее количество
узлов, либо требовать небольшую доработку полученного решения. Например, 
расширение адресного пространства. Тем не менее данные вопросы будут опущены 
по предположению, что целевая кластерная система будет содержать порядка 8-10 узлов.

Подзадачи:
\begin{itemize}
  \item Автоматическая установка \textmd{CentOS7};
  \item Настройка сетевой конфигурации;
  \item Установка и настройка Network File System (NFS)\cite{nfs};
  \item Установка и настройка Intel MPI.
\end{itemize}