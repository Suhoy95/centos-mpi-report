\chapter{Исспользуемые технологии}\label{technologies}

В данной разделе рассматриваются используемые программы и технологии,
используемые в данной работе и цель их использования.

\section{Kernel-based Virtual Machine (KVM) и libvirt}

Кластерная система -- это группа компьютеров, объединенных высокоскоростными
каналами связи, которая будет представлять для пользователя единый аппаратный
ресурс.

Для того, чтобы полноценно протестировать установку и настройку такой системы,
нам потребовалось несколько компьютеров и коммутатор(-ов) соединяющих их между
собой, что не очень целесообразно для тестовой задачи. Для этого в настоящий
момент времени удобно создавать такие системы с использованием виртуальных
машин и виртуальных сетей, которые будут эмулировать поведение реального
оборудования.

В Ubuntu рекомендуется использовать гипервизор (менеджер виртуальных машин)
\textmd{KVM}\cite{kvm} и библиотеку \textmd{libvirt}\cite{libvirt} в качестве инструментария
управления им.

\myImage{Virt-manager -- GUI-интерфейс для управления виртуальными машинами}{virt-manager}{virt-manager}

Также для простоты использования к данным инструментам можно поставить
\textmd{virt-manager}(Рис. \ref{virt-manager}) -- GUI-интерфейс, позволяющий
легко создавать виртуальные машины и управлять ими.

\section{Операционная система -- CentOS7}

Каждый узел (или нода) в нашей кластерной системе будет представлять
отдельный компьютер с установленной системой \textmd{CentOS7}\cite{centos},
основанной на пакетном менеджере \textmd{yum} и впервые выпущенной 14 мая 2004 года.
Эта операционная система основана на коммерческом проекте
\textmd{Red Hat Enterprise Linux}\cite{rhel} и совместима с ним.

Данная операционная система позволяет автоматизировать установку с помощью
\textmd{kickstart}-файла\cite{rhel-kickstart}, описывающего выбор вариантов,
которые предлагаются перед установкой системы.

\section{Network File System (NFS)}

Network File System (NFS) — протокол сетевого доступа к файловым системам,
первоначально разработан Sun Microsystems в 1984 году. За основу был взят протокол
вызова удалённых процедур (RPC). Позволяет подключать (монтировать) удалённые
файловые системы через сеть.

В нашей задаче данная технология потребуется для запуска MPI-программ, которые
подразумевают что запускаемая программа доступна на всех узлах системы.
Конечно, для этих целей можно просто провести копирование исполняемого файла
на все узлы, но постоянная работа таким образом является утомительной, что
в конечном итоге повлечет за собой создание скрипта или другого механизма
по автоматизации этого процесса. Чтобы избежать этого мы создадим каталог,
доступный через с помощью NFS всем узлам.

\section{MPI (Intel MPI)}

\textmd{Message Passing Interface} (MPI, интерфейс передачи сообщений) -- стандарт
программного интерфейса (API) для передачи информации, который позволяет
обмениваться сообщениями между процессами, выполняющими одну задачу.
Разработан Уильямом Гроуппом, Эвином Ласком (англ.) и другими\cite{mpi}.

Существует множество реализаций MPI. В данной работе мы будем разворачивать
\textmd{Intel MPI}\cite{intel-mpi}, в предположении что эта реализация
наибыстрейшая.

\section{Архитектура кластера}

Обсудим общую схему создаваемой системы. Краткое описание можно увидеть на
рисунке \ref{arch}. В текущем примере мы будем создавать кластер из 3-х узлов,
представляющих виртуальные машины с \textmd{CentOS7}, соединенных двумя сетями.

Первая сеть будет необходима для доступа в сеть Интернет и скачивания
требуемых rpm-пакетов. Вторая сеть будет использоваться для взаимодействия между
узлами и будет изолированной, поскольку протокол NFS подразумевает подключение
по доверенной сети.

Конфигурация первой сети:
\begin{verbatim}
  IP-пространство: 192.168.127.0/24
  Шлюз по-умолчанию: 192.168.127.1 (Рабочая станция, NAT до Интернета)
  DHCP-pool: 192.168.127.128 - 192.168.127.254
\end{verbatim}

Конфигурация второй сети:
\begin{verbatim}
  статическое IP-пространство: 10.20.30.0/24
\end{verbatim}

\clearpage

\myImage{Общая схема разворачиваемой системы}{arch}{arch}
Все узлы будут иметь названия хоста (node1/node2/node3). node1-узел
будет главным, а также будет размещать на себе NFS-сервер и предоставлять
папку /nfs двум другим узлам, которые будут NFS-клиентами.

Для того, чтобы обеспечить автоматическую установку системы
\textmd{CentOS}, потребуется обеспечить доступ к kickstart-файлам, которые
будут сгенерированны для каждого компьютера и располагаться на локальном
HTTP-сервере. Там же будет располагаться архив mpi.tgz с Intel MPI-установочником,
который будет установлен после основной установки системы.
